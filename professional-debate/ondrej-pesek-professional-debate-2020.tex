% ========================================================================
% 
% DOCTORAL THESIS
% Possibilities of use of artificial neural networks in work with spatial data
% 
% Ondřej Pešek
% 
% ========================================================================

\documentclass[
  12pt,         			% velikost základního písma je 12 bodů
  a4paper,      			% formát papíru je A4
  oneside,       			% Oboustranný tisk
  pdftex,				    % překlad bude proveden programem 'pdftex' do PDF
  english,
  %draft
]{report}       			% dokument třídy 'zpráva'


\newcommand{\Fbox}[1]{\fbox{\strut#1}}

\usepackage[english, czech]{babel}	% použití češtiny, angličtiny
\usepackage[utf8]{inputenc}			% kódování zdrojových souborů je UTF8

\usepackage[square,sort,comma,numbers]{natbib}

\usepackage{caption}
\usepackage{subcaption}
\usepackage{listings}

\usepackage[dvipsnames]{xcolor}
\definecolor{light-gray}{gray}{0.95}

\captionsetup{font=small}
\usepackage{enumitem}
\setlist{leftmargin=*} % bez odsazení

\makeatletter
\setlength{\@fptop}{0pt}
\setlength{\@fpbot}{0pt plus 1fil}
\makeatletter

\usepackage[dvips]{graphicx}   
\usepackage{color}
\usepackage{transparent}
\usepackage{wrapfig}
\usepackage{float} 

\usepackage{cmap}           
\usepackage[T1]{fontenc}    

\usepackage{textcomp}
\usepackage[compact]{titlesec}
\usepackage{amsmath}
\addtolength{\jot}{1em} 

\let\counterwithout\relax
\let\counterwithin\relax
\usepackage{chngcntr}
\counterwithout{footnote}{chapter}

\usepackage{acronym}

\usepackage[
    unicode,                
    breaklinks=true,        
    hypertexnames=false,
    colorlinks=true, % true for print version
    citecolor=black,
    filecolor=black,
    linkcolor=black,
    urlcolor=black
]{hyperref}         

\usepackage{url}
\usepackage{fancyhdr}
% \usepackage{algorithmic}
\usepackage{algorithm}
\usepackage{algcompatible}
\renewcommand{\ALG@name}{Pseudocode}% update algorithm name
\def\ALG@name{Pseudocode}

\usepackage[
  cvutstyle,          
  phd,
  english
]{thesiscvut}


\newif\ifweb
\ifx\ifHtml\undefined % mimo HTML.
    \webfalse
\else % v HTML.
    \webtrue
\fi 

\renewcommand{\figurename}{Obrázek}
\def\figurename{Obrázek}

\lstdefinestyle{python}{
   language=python,
   basicstyle={\footnotesize\ttfamily},
   keywordstyle=\color{blue}\ttfamily,
   stringstyle=\color{green}\ttfamily,
   commentstyle=\color{brown}\ttfamily,
   showstringspaces=false,
   morekeywords={True, False, sqrt}
}

%\renewcommand\lstlistingname{Pseudocode}
%\renewcommand*{\lstlistlistingname}{Content of pseudocodes}

\usepackage{dirtree}

\lstset{
	extendedchars=true,
	literate={á}{{\'a}}1}

\makeatletter
\newcommand\footnoteref[1]{\protected@xdef\@thefnmark{\ref{#1}}\@footnotemark}
\makeatother

\usepackage{tikz}
\usetikzlibrary{arrows.meta, shapes}
\tikzset{%
  >={Latex[width=1mm,length=1mm]},
  % Specifications for style of nodes:
            base/.style = {rectangle, rounded corners, draw=black,
                           minimum width=4cm, minimum height=1cm,
                           text centered, font=\sffamily},
  activityStarts/.style = {base, fill=blue!30},
       startstop/.style = {base, fill=red!30},
    activityRuns/.style = {base, fill=green!30},
    test/.style = {base, diamond, aspect=2, text width=8em, fill=yellow!30},
         process/.style = {base, minimum width=2.5cm, fill=orange!15,
                           font=\ttfamily},
}

\usepackage[justification=centering]{caption}

\newcommand\textstyleEmphasis[1]{\textit{#1}}
\newcommand\liststyleLi{%
\renewcommand\labelitemi{{\textbullet}}
\renewcommand\labelitemii{{\textbullet}}
\renewcommand\labelitemiii{{\textbullet}}
\renewcommand\labelitemiv{{\textbullet}}
}
\newcommand\liststyleLii{%
\renewcommand\labelitemi{{\textbullet}}
\renewcommand\labelitemii{{\textbullet}}
\renewcommand\labelitemiii{{\textbullet}}
\renewcommand\labelitemiv{{\textbullet}}
}

\newcommand\tab[1][1cm]{\hspace*{#1}}

% ========================================================================
% Definice informací o dokumentu
% ========================================================================

% název práce
\nazev{Possibilities of use of artificial neural networks in work with spatial data}
{Možnosti využití umělých neuronových sítí v práci s~prostorovými daty}

% jméno a příjmení autora
\autor{Ing. Ondřej}{Pešek}

% jméno a příjmení vedoucího práce včetně titulů
\garant{Prof. Aleš Čepek,~CSc.; Ing. Martin Landa,~PhD.}

% označení oboru studia
\oborstudia{Geomatics}{}

% označení ústavu
\ustav{Department of Geomatics}{}

% rok obhajoby
\rok{2020}

% měsíc obhajoby
\mesic{}

% místo obhajoby
\misto{Prague}

% abstrakt
\abstrakt 
{In recent years, the speed of technological progress in certain science fields is getting faster and faster. It is making it hard for other scientific areas to keep up with this tempo. One of the~exemplary relationships is the link between artificial neural network structures and the province of geomatics or remote sensing. New architectures of artificial neural network models are being published with an expeditious tempo and the common approach of the~remote sensing researchers is to use the most recent structures, without the~basic understanding of the background or relative performance. The goal of this thesis is to perform systematic research on the possibilities of use of chosen artificial neural network architectures on various selected use cases from the field of remote sensing.}
{Technologický vývoj v některých vědních odvětvích nabírá v posledních několika letech stále na rychlosti. A pro ostatní vědní obory je těžké držet s okolím tempo. Jedním takovým vztahem jsou struktury umělých neuronových sítí a obor geomatiky, případně dálkového průzkumu Země. Nové architektury modelů umělých neuronových sítí se objevují nebývalým kvapem, a běžný přístup výzkumníků dálkového průzkumu Země bývá ten, že si pro zpracovávání dat vybírají bez větší znalosti pozadí či porovnání nejnovější modely. Cílem této doktorské práce je systematicky prozkoumat a porovnat možnosti využití určených architektur umělých neuronových sítí na vybraných aplikačních tématech z prostředí dálkového průzkumu Země.}


% klíčová slova
\klicovaslova
{artificial neural networks, convolutional neural networks}
{umělé neuronové sítě, konvoluční neuronové sítě}

% ========================================================================
% Nastavení polí ve vlastnostech dokumentu PDF
% ========================================================================
\nastavenipdf

% začátek dokumentu
\begin{document}

\catcode`\-=12  % pro vypnutí aktivního znaku '-' používaného např. v \cline 

% aktivace záhlaví
\zahlavi

% předefinování vzhledu záhlaví
\renewcommand{\chaptermark}[1]{%
	\markboth{\MakeUppercase
	{%
	\thechapter.%
	\ #1}}{}}

% vysázení přebalu práce
%\vytvorobalku

% vysázení titulní stránky práce
\vytvortitulku

% Vysázení listu zadani
\stranka{}%
	%{\includegraphics[scale=0.7]{./pictures/zadanidp.pdf}}%\sffamily\Huge\centering\ }%ZDE VLOŽIT LIST ZADÁNÍ}%
	%{\sffamily\centering Z~důvodu správného číslování stránek}

% vysázení stránky s abstraktem
\vytvorabstrakt

% vysázení prohlaseni o samostatnosti
%\vytvorprohlaseni

% vysázení poděkování
\stranka{%nahore
       }{%uprostred
       }{%dole
       \sffamily
	\begin{flushleft}
		\large
		\MakeUppercase{Acknowledgement}
	\end{flushleft}
	\vspace{1em}
		%\noindent
	\par\hspace{2ex}
	{I would like to thank Martin Landa, not only for supervising my thesis and for the initial impulse in the direction to artificial neural networks and open source GIS generally, but monst erspecially for his doter endeurance when it cummes to my flishy-smolling prosterous occrazions.}
}

% vysázení obsahu
% \setcounter{tocdepth}{1}
\obsah

% vysázení seznamu obrázků
%\seznamobrazku

% vysázení seznamu tabulek
% \seznamtabulek

% vysázení seznamu ukázek kódu
%\cleardoublepage
%\thispagestyle{empty}
% \lstlistoflistings
%\newpage

% jednotlivé kapitoly
\chapter{Introduction}
\label{intro}

The future is arriving faster than we think. The technological progress is accelerating. Human knowledge is growing faster than ever before. If citing slightly lurid sources, according to Fuller's knowledge doubling curve \cite{knowledge-doubling-curve}, it took about 1500 years starting at the year 1 for the society amount of knowledge to double itself, then the doubling needed just about 100 years around the year 1900, and only 10 years around the year 1960; according to \cite{growth-forecast}, the knowledge doubled every 1.5 years around the year 2011. Besides many other aspects of such an information growth, it opens gates for technologies previously only dreamed up. Technologies like the artificial intelligence (\zk{AI}).

Artificial neural networks (\zk{ANN}s) became a term that is shaking the entire field of computer science. In the field of computer vision, it is especially their special type called convolutional neural networks (\zk{CNN}s) that is outperforming classical approaches to object detection and segmentation \cite{cnn-off-the=shelf}. Therefore, it should not be a surprise that \zk{ANN}s are widely used also in the field of remote sensing already since 2014 \cite{review-dl-rs-2017}, and their use outside and inside the field is just growing, as can be seen in figures \ref{fig-dl} and \ref{fig-rs-dl}.

\begin{figure}[h]
   \centering
	\includegraphics[width=0.6\linewidth]{./pictures/dl-papers.png}
	\caption[Papers on the use of DL]{Statistics for papers dealing with the use of deep learning in remote sensing: Citations listed in Google scholar for famous \zk{CNN} architecture \cite{cnn-classification}; arXiv listed publications in the categories \textit{cs} and \textit{stat} including the terms \textit{deep learning}, \textit{convolutional neural networks}, \textit{convolutional networks} or \textit{fully convolutional} and their share of all publications listed in the two categories; publications in selected Earth observation journals, searched for with the same terms as in arXiv, source: \cite{review-dl-eo}}
      \label{fig-dl}
\end{figure}

\begin{figure}[h]
   \centering
	\includegraphics[width=0.6\linewidth]{./pictures/dl-rs-papers.jpg}
	\caption[Papers on the use of DL in remote sensing]{Number of papers dealing with the use of deep learning in remote sensing per year, source: \cite{review-dl-rs-2017}}
      \label{fig-rs-dl}
\end{figure}

However, this quick development of \zk{ANN}s and \zk{CNN}s makes it harder and harder for other fields as remote sensing to keep up with the progress tempo. It results in the situation where when it comes to an \zk{ANN} architecture choose, researchers without the necessary background simply choose the newest one or the one with the best reported results, although these results could be obtained on completely different data in a completely different environment and the performance could therefore very much differ from the expected one. The goal of this thesis is to try to make complex, systematic research and review of the performance of chosen \zk{CNN} architectures on various selected use cases from the field of remote sensing, and hopefully serve as a valuable guidebook when it comes to \zk{CNN} applications in the field or comparison metrics when a new architecture is proposed.

Chapter \ref{motivation} will present research on the use of comparisons when it comes to papers dealing with \zk{CNN} models in the field of remote sensing and systematically investigate their senses of comparing the original input with the already existing work from different points of view to give them sufficient scientific context. The same systematical investigation will be done also for reviews of the use of \zk{CNN}s in the field.

% TODO: Delete info about the professional debate
Chapter \ref{use-cases} will present use cases selected as test tasks to evaluate chosen architectures. For purposes of the professional debate, only one use case is presented.

These chapters will be later followed by a chapter on selected architectures and methods and a chapter presenting the results of the experiments. For purposes of the professional debate, these are not present.
\chapter{Motivation}
\label{motivation}

In August 2020, an experiment was conducted - a query for papers with the below-described restrictions was done on selected academic database websites. The goal of the experiment was to research whether authors of papers dealing with the use of convolutional neural networks (\zk{CNN}) in the field of remote sensing are comparing results of proposed or used architectures with results of other architectures on the same task. This experiment is summarized in chapter \ref{top-papers}.

This experiment was immediately followed by an overview of the top \textit{review-type} papers - as something akin is the goal of this thesis, it could be taken as the~summary of the~current situation and research on work dealing with the~same problems. This is summarized in chapter \ref{situation}.

% possible titles:
% Top cited papers dealing with use of convolutional neural networks in the field of remote sensing
\section{Articles on CNNs in the field of remote sensing}
\label{top-papers}

\subsection{Web of Science}
\label{wos-papers}

The first used website was Web of Science\footnote{\url{www.webofknowledge.com}} (\zk{WoS}). The~following query restrictions were used:

\begin{itemize}
	\item \verb|Search string: cnn|. To focus only on articles dealing with \zk{CNN}s. The~term \verb|cnn| was preferred over the~term \verb|convolutional neural network| as in some papers, only the~abbreviated form was used.
	\item \verb|Publication years: 2020, 2019, 2018|. To focus only on recent publications.
	\item \verb|Web of Science categories: REMOTE SENSING|. To focus only on the~scien\-ti\-fic area of interest.
	\item \verb|Open Access: DOAJ Gold|. To focus only on articles and papers from sources listed on the~Di\-rectory of Open Access Journals\footnote{\url{www.doaj.org}} (\zk{DOAJ}).
\end{itemize}

\noindent The top five results from the query, when ordered by the~attribute \verb|Times Cited| and filtered as will be described below, were the following:

\begin{itemize}
	\item Evaluation of Different Machine Learning Methods and Deep-Learning Convolutional Neural Networks for Landslide Detection: 51 citations. \cite{landslide-evaluation}
	\item 3D Convolutional Neural Networks for Crop Classification with Multi-Temporal Remote Sensing Images: 50 citations. \cite{3d-cnn-crop}
	\item Geospatial Object Detection in High Resolution Satellite Images Based on Multi-Scale Convolutional Neural Network: 32 citations. \cite{object-detection-hrs-multi-scale}
	\item Hyperspectral Image Classification Using Convolutional Neural Networks and Multiple Feature Learning: 31 citations. \cite{hyperspectral-multiple-feat-cnn}
	\item Deformable Faster R-CNN with Aggregating Multi-Layer Features for Partially Occluded Object Detection in Optical Remote Sensing Images: 24 citations. \cite{deformable-faster-r-cnn}
\end{itemize}

Two results were filtered out. \cite{cnn-fusion-clouds} with 28 citations and \cite{cnn-fusion-hr-hsr} with 25 citations. They were filtered out only because they do not correspond with the~main focus of the~thesis on the classification task with object detection and semantic and instance segmentation, but dealt with an image fusion instead. 

The research starts very positively. The first result contains the phrase \textit{evaluation of different methods} already in the title and compares \zk{CNN}s with other popular machine learning (\zk{ML}) methods, namely support vector machine (\zk{SVM}) \cite{svm}, random forest (\zk{RF}) \cite{rf} and even with a simple artificial neural network (\zk{ANN}) architecture called multi-layer perceptron (\zk{MLP}) \cite{mlp} with a hidden layer of 30 neurons. Even the \zk{CNN} approach is diversified into two architectures - one of them comprising of five layers, the other one of seven layers. To make the results even more general, authors experiments also with five different input window sizes, compare the~use of spectral bands versus the~use of a combination of spectral bands and topographical ones and apply their models on two different datasets. Time consumption of different methods would be also valuable information on the comparison, yet this one is metric is not included in the article. It is apparent that authors of the paper came from the same impulse as this thesis, reading statements like \textit{"CNNs do not automatically outperform ANN, SVM and RF, although this is sometimes claimed. Rather, the performance of CNNs strongly depends on their design, i.e., layer depth, input window sizes and training strategies."} Or, in another place, \textit{"CNN will not automatically outperform other methods - as popular science articles and magazines may imply."} With felicitously set parameters, \zk{CNN}s outperformed other approaches, but their use was done under critical supervision and without trend-surfing shouts.

The second article, \textit{3D Convolutional Neural Networks for Crop Classification with Multi-Temporal Remote Sensing Images},  again experiments with different kernel sizes and other parameters, and compares the proposed 3D \zk{CNN} architecture with K-nearest neighbour (\zk{KNN}) \cite{knn}, principal component analysis (\zk{PCA}) \cite{pca}, \zk{SVM}, and an ad-hoc defined 2D \zk{CNN} structure. No time consumption analysis of different approaches was presented. Although the presented results seem to prove their claims that their proposed architecture performs better than the other ones, a more extensive research experimenting with more datasets and more advanced, state-of-the-art architectures could give such claims more solid position.

The comparison part of \textit{Geospatial Object Detection in High Resolution Satellite Images Based on Multi-Scale Convolutional Neural Network} is a bit more minimalistic. The proposed method is compared only with an architecture called Faster \zk{R-CNN} (region based convolutional neural network) \cite{faster-rcnn} and with the single shot multi-box detector (\zk{SSD}) \cite{ssd} without a more detailed description of inner parametrization or experiments with it. Authors use also only one dataset to test their method, so there is no evidence on how versatile the method is when it comes to the data greed. On the other hand, they report the time greed of the three used methods.

In \textit{Hyperspectral Image Classification Using Convolutional Neural Networks and Multiple Feature Learning}, authors go a~different way - to compare their architecture, they create two different architectures and show that the~proposed one is the~one performing the best. A comparison with any other well-known architecture or other \zk{ML} method is missing. The pro of this paper is the use of three different datasets.

24 citations reaching \textit{Deformable Faster R-CNN with Aggregating Multi-Layer Features for Partially Occluded Object Detection in Optical Remote Sensing Images} also proposes a new architecture, called \textit{deformable R-CNN}. Authors compare this model with \zk{SSD} and R-P-Faster \zk{R-CNN} \cite{rp-faster-rcnn} on three datasets. No comparison of time or memory requirements is included in the article.


% Convolution Neural Network Architecture Learning for Remote Sensing Scene Classification
% https://arxiv.org/abs/2001.09614

% Murthy, C.; Raju, P.; Badrinath, K. Classification of wheat crop with multi-temporal images: Performance of maximum likelihood and artificial neural networks. Int. J. Remote Sens. 2003, 24, 4871–4890. ?????????????????????????????

% Lin, H.; Shi, Z.; Zou, Z. Maritime Semantic Labeling of Optical Remote Sensing Images with Multi-Scale Fully Convolutional Network. ????????????

% WoS #2
% Geospatial Object Detection in High Resolution Satellite Images Based on Multi-Scale Convolutional Neural Network
% https://apps.webofknowledge.com/full_record.do?product=WOS&search_mode=GeneralSearch&qid=8&SID=C6dRsuZprCvu65HT7SF&page=1&doc=2&cacheurlFromRightClick=no
% superiority of the proposed method

% #3
% superior performances of the proposed framework

% #4
% Deformable Faster R-CNN with Aggregating Multi-Layer Features for Partially Occluded Object Detection in Optical Remote Sensing Images
% https://apps.webofknowledge.com/full_record.do?product=WOS&search_mode=GeneralSearch&qid=5&SID=C6pjuytIsVIIBFmwnOt&page=1&doc=4&cacheurlFromRightClick=no
% compared with SSD, R-P-Faster R-CNN, also compared on three datasets (NWPU, SORSI, HRRS), no comparison on time or memory usage

% #5
% Ship Detection Based on YOLOv2 for SAR Imagery
% https://apps.webofknowledge.com/full_record.do?product=WOS&search_mode=GeneralSearch&qid=5&SID=C6pjuytIsVIIBFmwnOt&page=1&doc=5&cacheurlFromRightClick=no
% compared with Faster R-CNN and not with YOLOv2 or YOLO, two datasets (both about ships), compared time needs
% "YOLOv2-reduced is best for real time object detection problem."

\subsection{Scopus}
\label{scopus-papers}

The second used website was Scopus\footnote{\url{www.scopus.com}}. The~following query restrictions were used:

\begin{itemize}
	\item \verb|Search string: "remote sensing" cnn|. To focus only on articles dealing with \zk{CNN}s in the area of remote sensing. The~term \verb|cnn| was preferred over the~term \verb|convolutional neural network| as in some papers, only the~abbreviated form was used.
	\item \verb|Publication years: 2020, 2019, 2018|. To focus only on recent publications.
	\item \verb|Access type: Open Access|. To focus only on articles in \textit{Scopus Gold Open Access}\footnote{\url{www.elsevier.com/open-access}}. It includes fully open journals, hybrid journals (authors pay a fee to make an article open access), open archives and articles with free promotional access.
\end{itemize}

\noindent The Scopus query is apparently - due to the~smaller flexibility when it comes to the~open access restrictions - more tolerant and includes articles filtered out from the~\zk{WoS} query. Also, because there is no scientific category \verb|remote sensing| in the~Scopus search engine, the~phrase was included in the~search phrase and more manual filtering was needed, as will be described below. The~top five results from the query, when ordered by the~attribute \verb|Cited by|, were the following:

\begin{itemize}
	\item A new deep convolutional neural network for fast hyperspectral image classification: 123 citations.  \cite{cnn-hs-class}
	\item Automatic ship detection in remote sensing images from google earth of complex scenes based on multiscale rotation Dense Feature Pyramid Networks: 97 citations. \cite{ship-rdfpn}
	\item Deep learning in remote sensing applications: A meta-analysis and review: 93 citations. \cite{dl-remote-sensing-review}
	\item Building extraction in very high resolution remote sensing imagery using deep learning and guided filters: 92 citations. \cite{vhr-building}
	\item Very Deep Convolutional Neural Networks for Complex Land Cover Mapping Using Multispectral Remote Sensing Imagery: 72 citations. \cite{very-deep-cnn-lc}
\end{itemize}

Four results were filtered out. \cite{dl-for-cv} with 251 citations, \cite{dl-lungs} with 87 citations, \cite{maoxian-landslide} with 74 citations, and \cite{state-of-the-art-dl} with 73 citations. They were filtered out only because they do not correspond with the~main focus of the~thesis on the classification task with object detection and semantic and instance segmentation in the field of remote sensing. 

The first article, \textit{A new deep convolutional neural network for fast hyperspectral image classification}, starts the~research again in a very positive way. Authors have compared their own architecture with the \zk{MLP} and three different \zk{CNN}s - a one-dimensional one, a two-dimensional one and a three-dimensional one. Although a~comparison with some popular architectures that just haven't been used on hyperspectral images or some classical \zk{ML} methods would be also interesting, the used ones are properly sourced to another research on hyperspectral image classification and authors underlined main differences between the proposed model and the ones used for evaluation. All experiments have been conducted on two datasets differing in the number of bands, pixel size of images, spatial resolution, and also in objects of classification. Authors also experiment with different patch sizes and - importantly - with the~number of samples per class.

\textit{Automatic ship detection in remote sensing images from google earth of complex scenes based on multiscale rotation Dense Feature Pyramid Networks} also does not compare the~proposed methodology with basic \zk{ML} approaches, but uses a lot of \zk{CNN} models to compare their architecture with - Faster \zk{R-CNN}, Feature pyramid network (\zk{FPN}) \cite{fpn}, rotation region proposal network (\zk{RRPN}) \cite{rrpn}, and rotational region convolutional neural network (\zk{R2CNN}) \cite{r2cnn}. However, as the~paper focuses only on the~ship detection, there is no experiment on other datasets.

The third most cited article on Scopus - \textit{Deep learning in remote sensing applications: A meta-analysis and review} - is a review coming from similar impulses as this thesis or partly \cite{landslide-evaluation} mentioned in chapter \ref{wos-papers}. The motivation is formulated in the sense that \textit{"it appears that a more systematic (i.e. quantitative) analysis is necessary to get a comprehensive and objective understanding of the applications of DL for remote-sensing analysis."}. Authors focus on many more fields than is the goal of this thesis, including image fusion, image registration, scene classification, object detection, land use and land cover classification, image segmentation, object-based image analysis (\zk{OBIA}), and other tasks. The~value of the~paper lies in its extensive research on what are the~most frequent targets of the~use of deep learning (\zk{DL}) for remote sensing, what are the~most frequent \zk{DL} models, spatial resolutions, application areas (urban, vegetation, etc.), average accuracies, common training datasets and even the~most used scientific terms in titles and abstracts of these papers. Although it is a high-class review and it works as a valuable overview about \zk{DL} stronger and weaker positions in the~field, authors have not conducted their own experiments, so only results reported in the~original papers are mentioned, if mentioned.

\textit{Building extraction in very high resolution remote sensing imagery using deep learning and guided filters} again proposes a new architecture and compares it to those of SegNet \cite{segnet}, fully convolutional network (\zk{FCN}) \cite{fcn}, a combination of \zk{CNN} and \zk{RF}, Multi-scale deep network \cite{hierarchical-labeling}, and a combination of \zk{CNN}, \zk{RF}, and conditional random fields (\zk{CRF}) \cite{hierarchical-labeling}, but authors do not explain how exactly are these models built, so it does not give the~reader any solid comparison. Especially terms like \zk{CNN} and \zk{FCN} are so general that they do not imply anything more than the basic approach. Time and memory needs are completely ignored in the~study and even though authors experimented on two datasets, both of them were of German cities and therefore very similar in the~content, although differing in used bands.

% Experimental results demonstrated that our methods were better than the other methods

The last reviewed article - \textit{Very Deep Convolutional Neural Networks for Complex Land Cover Mapping Using Multispectral Remote Sensing Imagery} - deals with land cover mapping using \zk{CNN}s with a focus on wetlands detection. Therefore it is not a surprise that they do not experiment with different datasets; however, it would give much more general overview of \zk{CNN} options in the wetland detection to use data from different parts of the world and not only from Newfoundland and Labrador, Canada. Besides that, the paper experiments with DenseNet-121 \cite{densenet}, Inception V3 \cite{inception}, VGG-16 \cite{vgg}, VGG-19, Xception \cite{xception}, \zk{ResNet}-50, and Inception \zk{ResNet} V2 \cite{inception-resnet}, and also with \zk{ML} methods of \zk{SVM} and \zk{RF}. Authors also use different patch sizes and report processing times.

\section{Reviews of CNNs in the field of remote sensing}
\label{situation}

\subsection{Web of Science}
\label{wos-reviews}

When compared with the query described in chapter \ref{wos-papers}, an extra parameter \verb|Document Types| was used to get only \textit{review-type} articles. Therefore, the following restrictions were used for the~query:

\begin{itemize}
	\item \verb|Search string: cnn|. To focus only on articles dealing with \zk{CNN}s. The~term \verb|cnn| was preferred over the~term \verb|convolutional neural network| as in some papers, only the~abbreviated form was used.
	\item \verb|Publication years: 2020, 2019, 2018|. To focus only on recent publications.
	\item \verb|Web of Science categories: REMOTE SENSING|. To focus only on the~scien\-ti\-fic area of interest.
	\item \verb|Open Access: DOAJ Gold|. To focus only on articles and papers from sources listed on the~Di\-rectory of Open Access Journals\footnote{\url{www.doaj.org}} (\zk{DOAJ}).
	\item \verb|Document Types: REVIEW|. To focus only on reviews.
\end{itemize}

\noindent There were only five results for the~described query and one of them - \textit{Spatiotemporal Image Fusion in Remote Sensing} \cite{review-st-fusion} with 14 citations - was filtered out as it dealt with the~image fusion and not with the~classification. The rest, when ordered by the~attribute \verb|Times Cited|, were the following:

\begin{itemize}
	\item Review and Evaluation of Deep Learning Architectures for Efficient Land Cover Mapping with UAS Hyper-Spatial Imagery: A Case Study Over a Wetland: 3 citations. \cite{review-dl-wetlands}
	\item UAV-Based Structural Damage Mapping: A Review: 3 citations. \cite{uav-building-damages}
	\item Object Detection and Image Segmentation with Deep Learning on Earth Observation Data: A Review-Part I: Evolution and Recent Trends: 1 citation. \cite{review-dl-eo}
	\item Object Detection and Image Segmentation with Deep Learning on Earth Observation Data: A Review-Part I: Evolution and Recent Trends: 1 citation. \cite{review-dl-eo}
	\item Geographic Object-Based Image Analysis: A Primer and Future Directions: 0 citation. \cite{geobia}
\end{itemize}

As \cite{very-deep-cnn-lc}, \textit{Review and Evaluation of Deep Learning Architectures for Efficient Land Cover Mapping with UAS Hyper-Spatial Imagery: A Case Study Over a Wetland} focuses on wetlands. And although authors evaluate and describe a pleasurable amount of architectures including SegNet, U-Net \cite{u-net}, DenseNet, DeepLab V3+ \cite{deeplab}, Pyramid scene parsing network (\zk{PSPNet}) \cite{pspnet}, and an architecture called MobileU-Net (a combination of MobileNet \cite{mobilenet} and U-Net) and dedicate a lot of space to used evaluation metrics, their findings would be even more valuable and universal if tested on more than one dataset. An application of at least one of the~common \zk{ML} methods would also give the~article an added value in the form of a~well-understood relative comparison.

\textit{UAV-Based Structural Damage Mapping} does not focus primarily on \zk{CNN}s, but as they constitute a big part of UAV-based damage mapping, there is a lot of reviewing of them in the~article. However, the~article works more as an overview of the evolution of different approaches in the topic. Therefore, no novel experiments were conducted and when some results are mentioned, they are only copied from original papers; papers, where these methods could have been used on different data with different results.

The same approach - major focus on an overview and the~evolution - is also in \textit{Object Detection and Image Segmentation with Deep Learning on Earth Observation Data: A Review-Part I} (part two is as of August 14, 2020 not yet published). Authors present a comprehensive overview of \zk{DL} architectures commonly or less commonly used on Earth observation (\zk{EO}) data, but do not conduct their own experiments; instead, they report results based on the~\zk{MS-COCO} dataset (Microsoft-Common Objects in Context) \cite{coco}, a dataset with contents very different than those of \zk{EO} data.

The main focus of \textit{Geographic Object-Based Image Analysis: A Primer and Future Directions} is on the geographic object-based analysis (\zk{GEOBIA}) and not \zk{CNN}s; however, a not negligible part of the~review deals with \zk{CNN}s and \zk{CNN}s - called geographic object-based convolutional neural networks (\zk{GEOCNN}s) - are proposed as one of the~most promising future directions of \zk{GEOBIA}. But when it comes to the~chapter \textit{Accuracies of GEOCNN Methods Versus Conventional GEOBIA}, no experiments are conducted and the~entire report is reduced to citations with trifling claims without numbers like \textit{"method following Approach 3 resulted in higher thematic accuracy than per-pixel classification with patch-based CNNs and FCNs"} or \textit{"method following Approach 2 resulted in higher segmentation accuracy than GEOBIA"}.

\subsection{Scopus}
\label{scopus-reviews}

\chapter{Use cases}
\label{use-cases}

The best comparison is the one using things readers already know. The best comparison is the one from the real world and praxis.

% TODO: After the professional debate, delete the note about it
As the aim of this thesis is to evaluate the performance of different \zk{CNN} models on different tasks from the field of remote sensing and make an order in \textit{why is an~architecture X used on Y}, use cases chosen for experiments should correspond with tasks normal remote sensing researchers could face. This chapter presents a~research on these chosen use cases (currently - for the purpose of the professional debate - the phrase \textit{use cases} should be actually used in a singular form as there is only one use case so far).

% vysázení seznamu zkratek

\begin{seznamzkratek}{ABCDE}

	\novazkratka{AI}
	      {AI}
	      {\qquad Artificial intelligence}

	\novazkratka{ANN}
	      {ANN}
	      {\qquad Artificial neural network}

	\novazkratka{ARVI}
	      {ARVI}
	      {\qquad Atmospherically resistant vegetation index}

	\novazkratka{CD-CNN}
	      {CD-CNN}
	      {\hspace{5.0mm} Contextual deep convolutional neural network}

	\novazkratka{CI}
	      {CI}
	      {\qquad Crust index}

	\novazkratka{CLC}
	      {CLC}
	      {\qquad CORINE land cover}

	\novazkratka{CNN}
	      {CNN}
	      {\qquad Convolutional neural network}

	\novazkratka{CNN-PPF}
	      {CNN-PPF}
	      {\hspace{2.9mm} Convolutional neural network using pixel-pair features}

	\novazkratka{CORINE}
	      {CORINE}
	      {\hspace{5.0mm} Coordination of information on the environment}

	\novazkratka{CRF}
	      {CRF}
	      {\qquad Conditional random fields}

	\novazkratka{CRNN}
	      {CRNN}
	      {\qquad Convolutional recurrent neural network}

	\novazkratka{DBI}
	      {DBI}
	      {\qquad Davies-Bouldin index}

	\novazkratka{DL}
	      {DL}
	      {\qquad Deep learning}

	\novazkratka{DOAJ}
	      {DOAJ}
	      {\qquad Directory of Open Access Journals}

	\novazkratka{DSM}
	      {DSM}
	      {\qquad Digital surface model}

	\novazkratka{EEA}
	      {EEA}
	      {\qquad European Environment Agency}

	\novazkratka{EO}
	      {EO}
	      {\qquad Earth observation}

	\novazkratka{EVI}
	      {EVI}
	      {\qquad Enhanced vegetation index}

	\novazkratka{FCN}
	      {FCN}
	      {\qquad Fully convolutional network}

	\novazkratka{FPN}
	      {FPN}
	      {\qquad Feature pyramid network}

	\novazkratka{GEOBIA}
	      {GEOBIA}
	      {\quad \hspace{1mm} Geographic object-based image analysis}

	\novazkratka{GEOCNN}
	      {GEOCNN}
	      {\hspace{3.6mm} Geographic object-based convolutional neural network}

	\novazkratka{GMM}
	      {GMM}
	      {\qquad Gaussian mixture model}

	\novazkratka{GRSS}
	      {GRSS}
	      {\qquad Geoscience and remote sensing society}

	\novazkratka{HSI}
	      {HSI}
	      {\qquad Hue, saturation, intensity}

	\novazkratka{IEEE}
	      {IEEE}
	      {\qquad Institute of electrical and electronics engineers}

	\novazkratka{KNN}
	      {KNN}
	      {\qquad K-nearest neighbour}

	\novazkratka{L-GCNN}
	      {L-GCNN}
	      {\quad $\-$  $\-$ Learnable-gated convolutional neural network}

	\novazkratka{LapSVM}
	      {LapSVM}
	      {\quad $\-$ $\-$ Laplacian support vector machine}

	\novazkratka{LiDAR}
	      {LiDAR}
	      {\qquad Light detection and ranging}

	\novazkratka{ML}
	      {ML}
	      {\qquad Machine learning}

	\novazkratka{MLP}
	      {MLP}
	      {\qquad Multi-layer perceptron}

	\novazkratka{MS-COCO}
	      {MS-COCO}
	      {$\>$ $\-$ Microsoft-Common Objects in Context}

	\novazkratka{mUnet}
	      {mUnet}
	      {\qquad Modified U-Net}

	\novazkratka{NAIP}
	      {NAIP}
	      {\qquad National agricultural imagery program}

	\novazkratka{NDSM}
	      {NDSM}
	      {\qquad Normalized digital surface model}

	\novazkratka{NDVI}
	      {NDVI}
	      {\qquad Normalized differential vegetation index}

	\novazkratka{OBIA}
	      {OBIA}
	      {\qquad Object-based image analysis}

	\novazkratka{PCA}
	      {PCA}
	      {\qquad Principal component analysis}

	\novazkratka{PGM}
	      {PGM}
	      {\qquad Parameterized gate module}

	\novazkratka{PL-SSDL}
	      {PL-SSDL}
	      {\quad$\-\>$ Semi-supervised deep learning using pseudo labels}

	\novazkratka{PSPNet}
	      {PSPNet}
	      {\qquad Pyramid scene parsing network}

	\novazkratka{R-CNN}
	      {R-CNN}
	      {\qquad Region-based convolutional neural network}

	\novazkratka{R2CNN}
	      {R\textsuperscript{2}CNN}
	      {\qquad Rotational region convolutional neural network}

	\novazkratka{RBF}
	      {RBF}
	      {\qquad Radial basis function}

	\novazkratka{ResNet}
	      {ResNet}
	      {\qquad Residual network}

	\novazkratka{RNN}
	      {RNN}
	      {\qquad Recurrent neural network}

	\novazkratka{RRPN}
	      {RRPN}
	      {\qquad Rotation region proposal network}

	\novazkratka{RF}
	      {RF}
	      {\qquad Random forest}

	\novazkratka{SAR}
	      {SAR}
	      {\qquad Synthetic aperture radar}

	\novazkratka{SAVI}
	      {SAVI}
	      {\qquad Soil adjusted vegetation index}

	\novazkratka{SDA}
	      {SDA}
	      {\qquad Stacked denoising autoencoder}

	\novazkratka{SBS}
	      {SBS}
	      {\qquad Sequential backward selection}

	\novazkratka{SFS}
	      {SFS}
	      {\qquad Sequential forward selection}

	\novazkratka{SS-LapSVM}
	      {SS-LapSVM}
	      {$\-$ Spatio-spectral Laplacian support vector machine}

	\novazkratka{SSD}
	      {SSD}
	      {\qquad Single shot multi-box detector}

	\novazkratka{SVM}
	      {SVM}
	      {\qquad Support vector machine}

	\novazkratka{TSVM}
	      {TSVM}
	      {\qquad Transductive support vector machine}

	\novazkratka{VARI}
	      {VARI}
	      {\qquad Visual atmospheric resistance index}

	\novazkratka{WoS}
	      {WoS}
	      {\qquad Web of Science}
	      
\end{seznamzkratek}

% literatura
\nocite{*}
\def\refname{References}
\bibliographystyle{mystyle}
\bibliography{literatura}


% začátek příloh
%\def\figurename{Figure}%
%\prilohy

% vysázení seznamu příloh
%\seznampriloh

% Vložení souboru s přílohami
%\include{prilohy}

% konec dokumentu
\end{document}
